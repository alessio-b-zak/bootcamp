\documentclass{beamer}
\usetheme{Boadilla}
\usepackage{amssymb}


\def\rnum{\mathbb{R}}

\title{Convex Optimisation}
\author{Alessio Zakaria}
\date{}
\begin{document}
\begin{frame}
    \titlepage
\end{frame}

\begin{frame}
    \frametitle{Convex Optimisation?}
    \begin{itemize}
    \item What is it?
        \begin{itemize}
        \item Finding the maxima / minima of convex functions over convex sets
        with respect to convex or affine constraints
        \end{itemize}
    \item Why do we care?
        \begin{itemize}
            \item Convex functions display many theoretical properties that are
                suited to optimisation
            \item Solving convex optimisation allows you to place a lower bound
                on non-convex functions
            \item Many real world optimisations are convex
        \end{itemize}
    \end{itemize}
\end{frame}
\begin{frame}
    \frametitle{Convexity}
    A set $C \subseteq \mathbb{R}$ is convex if $\forall \,  x,  y \in C, \,
    \forall \, \theta \in [0, \, 1]$
    \begin{align*}
        \theta x + (1-\theta)y \in C
    \end{align*}
    i.e. all points on the line segment between $x$ and $y$ lie in C
\end{frame}

\begin{frame}
    \frametitle{Convex Functions}
    A function $f : C \subseteq \mathbb{R}^{n} \rightarrow \rnum^{m}$ is convex
    on some convex set $C$ if $\forall \, x, \, y \in C, \forall \, \lambda \in
    [0,1]$
    \begin{align*}
        f(\lambda x + (1 - \lambda)y) \leq f(\lambda x) + f((1-\lambda)y)
    \end{align*}
    \todo{picture of convexity or further explanation}
\end{frame}
\begin{frame}
    \frametitle{Minima and Maxima}
    $f: S \subseteq \rnum^{n} \rightarrow \rnum$ has a global minimum (maximum
    resp.) $x^{\star}$ if:
    \begin{align*}
        \forall \, x \in S, \, \, f(x^{\star}) \leq f(x), \hspace{2mm}  (f(x^{\star}) \geq f(x))
    \end{align*}
    \\~\\
    $f$ has a local minimum (maxmum resp.) around $x^{\star}$ if $\exists \, R \in
     \rnum$ \text{ such that }
    \begin{align*}
        \forall x \, \in B(x^{\star}, R), \, \, f(x^{\star}) \leq f(x), \hspace{2mm} 
    \end{align*}<++>
\end{frame}<++>
\end{document}

