\documentclass[a4paper,10pt]{article}

\usepackage{listings}
\usepackage[utf8]{inputenc}

\usepackage{mathtools}
\usepackage{amsthm}


\let\oldnorm\norm   % <-- Store original \norm as \oldnorm
\let\norm\undefined % <-- "Undefine" \norm
\DeclarePairedDelimiter\norm{\lVert}{\rVert}

\def\parw{\ensuremath{\textbf{w}}}

\def\exp{\ensuremath{\mathbb{E}}}
\def\sinn{\text{sin}}
\def\coss{\text{cos}}

\def\phix{\ensuremath{\phi(\textbf{X})}}

\usepackage[margin=1.00in]{geometry}
\usepackage{bm}
\usepackage{cite}
\newcommand{\bigCI}{\mathrel{\text{\scalebox{1.07}{$\perp\mkern-10mu\perp$}}}}

\def\bolxi{\ensuremath{\textbf{x}_{i}}}
\usepackage{titling}
\usepackage{indentfirst}
\usepackage[textwidth=3.7cm]{todonotes}
\setlength{\marginparwidth}{3.7cm}
\usepackage{etoolbox}
\lstset{breaklines}
\usepackage{amssymb}

\usepackage{amsmath}


\def\rnum{\mathbb{R}}


\newtheorem*{definition}{Definition}
\newtheorem*{theorem}{Theorem}



\usepackage[compact]{titlesec}         % you need this package
\titlespacing{\section}{1pt}{1pt}{1pt} % this reduces space between (sub)sections to 0pt, for example
\titlespacing{\subsection}{0pt}{0pt}{0pt} % this reduces space between (sub)sections to 0pt, for example
\titlespacing{\subsubsection}{0pt}{0pt}{0pt} % this reduces space between (sub)sections to 0pt, for example
\AtBeginDocument{%                     % this will reduce spaces between parts (above and below) of texts within a (sub)section to 0pt, for example - like between an 'eqnarray' and text
    \setlength\abovedisplayskip{8pt}
    \setlength\belowdisplayskip{0pt}
}

\setlength{\droptitle}{-20mm} % This is your set scream
\setlength{\parindent}{2.8em}
\setlength{\parskip}{1.8em}
\setlength{\parindent}{1.8in}

\date{}
\author{Alessio Zakaria}
\title{Preliminary Definitions\vspace{-30mm}}
\begin{document}
\maketitle
\begin{definition}
A matrix, $A \in \rnum^{m \times n}$ is positive semidefinite if:
\begin{align*}
    \forall     x \in \rnum^{n}, \, x^{T}Ax \geq 0
\end{align*}
\end{definition}
\begin{definition}
A function $f$ is affine if
\begin{align*}
    f(v) = Av + b
\end{align*} for some matrix $A$ and some vector $b$.
\end{definition}

\begin{definition}
    For some minimisation problem with objective function $f: \rnum^{n} \rightarrow \rnum$ and inequality
    constraints $g_{i}: \rnum^{n} \rightarrow \rnum$ and equality constraints
    $h_{j}: \rnum^{n} \rightarrow \rnum$ that are continuously differentiable at the optimal solution $x^{\star}$ the KKT conditions are (providing some constraints hold):
    $\exists \, \, \mu_{i} \text{ and }  \lambda_{i}\text{ such that}$
    \begin{align*}
        &\nabla f(x^{\star}) + \sum\limits_{i=1}^{m}\mu_{i} \nabla g_{i}(x^{\star}) + \sum\limits_{j=1}^{l}\lambda_{j}\nabla h_{j}(x^{\star}) = 0 \,\,\,\, \,\, \text{(stationarity)}\\
        &g_{i}(x^{\star}) \leq 0 \,\,\,\,\,\,\,\,\, \text{(Primal Feasibility)}\\
        &h_{j}(x^{\star}) = 0 \,\,\,\,\,\,\,\,\,\,\, \text{(Primal Feasibility)}\\
        &\mu_{i} \geq 0 \,\,\,\,\,\,\,\,\,\,\, \text{(Dual Feasibility)}\\
        &\mu_{i}g_{i}(x^{\star}) = 0 \,\,\,\,\,\,\,\,\,\,\, \text{(Complementary Slackness)}
    \end{align*}

    \label{a}
\end{definition}


\begin{definition}
    A point $x \in \rnum^{n}$ is a critical point of some $f : \rnum^{n \times m} \rightarrow \rnum$ if:
    \begin{align*}
        \nabla f(x) = \textbf{0}
    \end{align*}
    \label{<+label+>}
\end{definition}

\begin{theorem}
    Fermat's Theorem (critical points): If $x \in \rnum^{n}$ is a local minima of some function $f : \rnum^{n} \rightarrow \rnum$ then:
    \begin{align*}
        \nabla f(x) = \textbf{0}
    \end{align*}
    \label{<+label+>}
\end{theorem}


\begin{definition}
    The ball of size $R > 0$ around a point $x \in \rnum^{n}$, written $B(x, R)$, is the set of all points $y \in \rnum^{n}$ such that:
    \begin{align*}
        \norm{y - x} < R
    \end{align*}
    \label{j}
\end{definition}

\begin{definition}
    An point, $x \in \rnum^{n}$ of some set $C \subseteq \rnum^{n}$ is an interior point if there exists an $R > 0 $ such that:
    \begin{align*}
        B(x, R) \subseteq C
    \end{align*}
    \label{<+label+>}
\end{definition}

\begin{theorem}
    $2^{nd}$ Order Taylor's Approximation: If $f : U \rightarrow \rnum$ is a
    twice differentiable function over an open set $U \subseteq \rnum^{n}$ and
    $x \in U$ select some $r > 0$ such that $B(x, r) \subseteq U$. Then for any
    $y \in B(x,r)$
    \begin{align*}
        f(y) = f(x) + \nabla f(x)^{T}(y-x) + \frac{1}{2}(y-x)^{2}\nabla^{2}f(x)(y-x) + o(\norm{y-x}^{2})
    \end{align*}
    \label{k}
\end{theorem}


\begin{definition}
    A linear program is an optimisation of the form:
    \begin{align*}
        \underset{x}{\text{maximise }} \textbf{c}^{T}\textbf{x} \\
        \text{such that } \textbf{A}\textbf{x} \leq \textbf{b} \\
        \text{and } \textbf{x} \geq \textbf{0}
    \end{align*}
    $A \in \rnum^{m \times n}$, $c \in \rnum^{n}$, $b \in \rnum^{m}$ and $x \in rnum^{n}$
    \label{t}
\end{definition}



\end{document}
